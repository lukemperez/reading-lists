% original template by svmiller
% modifications my joshuascriven

\documentclass[10pt,]{article}
\usepackage[margin=0.85in]{geometry}
\usepackage[hyphens]{url}
\usepackage{float, bbding}
\usepackage{wrapfig}
\newcommand*{\authorfont}{\fontfamily{phv}\selectfont}
\usepackage[]{mathpazo}
\usepackage{abstract}
\renewcommand{\abstractname}{}    % clear the title
\renewcommand{\absnamepos}{empty} % originally center
\newcommand{\blankline}{\quad\pagebreak[2]}

\providecommand{\tightlist}{%
  \setlength{\itemsep}{0pt}\setlength{\parskip}{0pt}}
\usepackage{longtable,booktabs}

\usepackage{parskip}
\usepackage{titlesec}
\titlespacing\section{0pt}{12pt plus 4pt minus 2pt}{6pt plus 2pt minus 2pt}
\titlespacing\subsection{0pt}{12pt plus 4pt minus 2pt}{6pt plus 2pt minus 2pt}

\titleformat*{\subsubsection}{\normalsize\itshape}

\usepackage{titling}
\setlength{\droptitle}{-.25cm}

%\setlength{\parindent}{0pt}
%\setlength{\parskip}{6pt plus 2pt minus 1pt}
%\setlength{\emergencystretch}{3em}  % prevent overfull lines

\usepackage[T1]{fontenc}
\usepackage[utf8]{inputenc}

\usepackage{fancyhdr}
\pagestyle{fancy}
\usepackage{lastpage}
\renewcommand{\headrulewidth}{0.3pt}
\renewcommand{\footrulewidth}{0.0pt}
\lhead{}
\chead{}
\rhead{\footnotesize INR3933: International Organizations \textbar{}
(Online Instruction) -- Spring 2021}
\lfoot{}
\cfoot{\small \thepage/\pageref*{LastPage}}
\rfoot{}

\fancypagestyle{firststyle}
{
\renewcommand{\headrulewidth}{0pt}%
   \fancyhf{}
   \fancyfoot[C]{\small \thepage/\pageref*{LastPage}}
}

%\def\labelitemi{--}
%\usepackage{enumitem}
%\setitemize[0]{leftmargin=25pt}
%\setenumerate[0]{leftmargin=25pt}


\makeatletter
\@ifpackageloaded{hyperref}{}{%
\ifxetex
  \usepackage[setpagesize=false, % page size defined by xetex
              unicode=false, % unicode breaks when used with xetex
              xetex]{hyperref}
\else
  \usepackage[unicode=true]{hyperref}
\fi
}
\@ifpackageloaded{xcolor}{
    \PassOptionsToPackage{usenames,dvipsnames}{xcolor}
}{%
    \usepackage[usenames,dvipsnames]{xcolor}
}
\makeatother
\hypersetup{breaklinks=true,
            bookmarks=true,
            pdfauthor={ ()},
            pdfkeywords = {},
            pdftitle={INR3933: International Organizations \textbar{}
(Online Instruction)},
            colorlinks=true,
            citecolor=Sepia,
            urlcolor=Sepia,
            linkcolor=Sepia,
            pdfborder={0 0 0}}
\urlstyle{same}  % don't use monospace font for urls


\setcounter{secnumdepth}{0}

\usepackage{longtable}




\usepackage{graphicx}
\usepackage{setspace}

% Remove the author field and the space associated
% from the definition of maketitle!
\makeatletter
\renewcommand{\@maketitle}{
\newpage
 \null
 \vskip 2em%
 \begin{center}%
  {\LARGE \@title \par}%
  \@date
  
    \includegraphics[width=30mm,scale=0.5]{seal}
   \end{center}%
 \par} \makeatother
 

\title{INR3933: International Organizations \textbar{} (Online
Instruction)}
\author{Lecturer: Joshua Scriven}
\date{Spring 2021}




%%%%% tables that span multiple pages
\usepackage{longtable}

%%%%color in tables
\usepackage{xcolor,colortbl}

%%%%for resizing tables
\usepackage{adjustbox}

\newcommand{\mc}[2]{\multicolumn{#1}{c}{#2}}
\definecolor{Gray}{gray}{0.85}
\definecolor{LightCyan}{rgb}{0.88,1,1}


\begin{document}

		\maketitle
	

		\thispagestyle{firststyle}

\begin{center}
\begin{tabular}{llccll}
\hline
\rowcolor{Gray}
Instructor: & Joshua Scriven &          \hspace{3cm}           &
Email: & \href{mailto:jscriven@fsu.edu}{\nolinkurl{jscriven@fsu.edu}}\\

Classroom: &  Zoom link on Canvas & &
Class Time: & T/R 12:00--13:30 (see below)\\
Office: & Zoom link on Canvas & &
Office Hours: & T/R 13:30--14:30 (by appointment)\\
\hline
\end{tabular}

\begin{minipage}{5.2in}
  \begin{flushleft}
    {\color{Gray}{\RectangleBold}} ~{\footnotesize ``{[}T{]}he blood of
millions of men, numberless and unprecedented sufferings, useless
slaughter, and frightful ruin, are the sanction of the covenant which
unites you in a solemn pledge which must change the future history of
the world.'' --Address of his Holiness Pope Paul VI to the United
Nations (Oct.~4, 1965)}\newline
  \end{flushleft}
\end{minipage}


\end{center}

\vspace{2mm}


\hypertarget{course-description}{%
\section{Course Description}\label{course-description}}

In this course, students will use a combination of lectures, live
debate, and research projects to cover topics in the study of
international organizations (IOs) distributed across seven modules which
collectively answer four questions:

\begin{enumerate}
\def\labelenumi{\arabic{enumi}.}
\tightlist
\item
  What are IOs and why do they exist?
\item
  Are IOs independent actors in international relations?
\item
  Do IOs have a positive, negative, or null effect on the behavior of
  international actors?
\end{enumerate}

\hypertarget{learning-objectives}{%
\section{Learning Objectives}\label{learning-objectives}}

This one-semester course is intended to:

\begin{enumerate}
\def\labelenumi{\arabic{enumi}.}
\tightlist
\item
  introduce the fundamental concepts surrounding IOs and their
  relationship with other international actors;
\item
  improve your understanding of the historical and potential role of
  international organizations;
\item
  help you analyze the interplay between IO delegation and state
  sovereignty;
\item
  examine circumstances where IOs have contributed to stability and
  prosperity in the international community, while accounting for
  instances where they have been a source of inefficiency and
  inequality;
\item
  discuss whether IOs can be decribed as `necessary' for international
  governance.
\end{enumerate}

\hypertarget{required-textbook}{%
\section{Required Textbook}\label{required-textbook}}

Hurd, Ian (2020).
\emph{International organizations: politics, law, practice}. 4th ed.
Cambridge University Press.

\hypertarget{attendance}{%
\section{Attendance}\label{attendance}}

We meet for class every Thursday and every other Tuesday of the
semester, following the schedule available on Canvas. Discussion
meetings, which occur every other Thursday (as indicated in the course
schedule) last for 1.5 hours, in contrast to lecture meetings, which
last 1.25 hours. Attendance is not required, but groups are encouraged
to indicate absences when evaluating members who have not contributed
\emph{and} are not present for group work, such as in-class discussions.

When applicable, excused absences include documented illness, deaths in
the family and other documented crises, call to active military duty or
jury duty, religious holy days, and official University activities.
These absences will be accommodated in a way that does not arbitrarily
penalize students who have a valid excuse. Consideration will also be
given to students whose dependent children experience serious illness.

\hypertarget{grades}{%
\section{Grades}\label{grades}}

\hypertarget{composition}{%
\subsection{Composition}\label{composition}}

Your final course grade is a combination of scores according to the
table shown below.

\hypertarget{topical-discussions-group-grade-individually-weighted}{%
\subsubsection{Topical Discussions (Group Grade, Individually
Weighted)}\label{topical-discussions-group-grade-individually-weighted}}

Students will participate in each module's discussion, live on Zoom on
the Thursdays indicated. It is expected that both groups and individual
students come prepared to give responses during these discussions, as
the discussion questions (which pertain to the module's subject matter)
will be provided beforehand. The grade a student receives for his/her
discussion response participation will be based on the professor's
evaluation of his/her group's contribution to the discussion, and
weighted (adjusted) by their group members' peer evaluations of their
contributions to the same, \textbf{\emph{which must be completed within
the last 5 minutes of every discussion meeting, before leaving the Zoom
session}}.

\hypertarget{group-research-project-group-grade-individually-weighted}{%
\subsubsection{Group Research Project (Group Grade, Individually
Weighted)}\label{group-research-project-group-grade-individually-weighted}}

Each group will be assigned a research project on an international
organization. This will be the international organization that shares
the name of their group, unless the group unanimously votes otherwise.
There are two parts to the project.

\begin{itemize}
\tightlist
\item
  \textbf{Part 1} requires groups to submit a set of discussion
  questions (i.e., questions for which there are valid, divergent
  responses). Groups will provide the class with questions for the
  topical discussion by the Tuesday prior to that discussion.
\item
  \textbf{Part 2} requires the creation of a PDF slideshow which:

  \begin{itemize}
  \tightlist
  \item
    introduces the class to the organization and analyzes its historical
    origins and institutional structure, and
  \item
    evaluates whether and how the organization matters in world
    politics, using examples from the text, lecture, or personal
    research. Groups will present on their research during the
    applicable topical discussion. Presentations should last between
    10-15 minutes.
  \end{itemize}
\end{itemize}

The grade a student receives for this assignment will be based on the
professor's evaluation of their group's delivery of both the presenation
and the discussion, which is then weighted (adjusted) by their group
members' peer evaluations of their contributions to the same,
\textbf{\emph{which must be completed within the last 5 minutes of every
discussion meeting, before leaving the Zoom session.}}

\hypertarget{reading-guides-individual-grade}{%
\subsubsection{Reading Guides (Individual
Grade)}\label{reading-guides-individual-grade}}

For each module, students complete this individual-work assignment as a
way to explore more complex concepts presented in the reading materials,
which is helpful for a proper undertanding of the semester's coursework
as a whole. A reading guide consists of a set of questions directly
related to a module's readings. Students are, therefore,neither
encouraged nor expected to turn to outside resources to complete the
reading guides.

\hypertarget{final-exam-individual-grade}{%
\subsubsection{Final Exam (Individual
Grade)}\label{final-exam-individual-grade}}

The final exam is a multiple choice quiz that is cumulative, open-book,
and comprises questions drawn from the lecture slides and assigned
readings (and therefore, the reading guides as well). It is timed, and
completed individually during finals week.

\hypertarget{scales}{%
\subsection{Scales}\label{scales}}

This course uses the grading scale displayed here. Where they exist, due
dates should be strictly adhered to. Assignments will receive a
deduction of 5 percent for each calendar day they are late. Students are
expected to complete missing assignments before seeking any extra credit
(which is always offered to the entire class). Extra credit will not be
assigned within the last two weeks of the semester.

\begin{table}[!h]
\caption{\label{tab:unnamed-chunk-2}Grade Composition and Grading Scale}

\centering
\begin{tabular}[t]{ll}
\toprule
Component & \%\\
\midrule
Final Exam & 20\\
Research Project (Presentation) & 20\\
Research Project (Discussion) & 10\\
Reading Guides & 20\\
Topical Discussions & 30\\
\addlinespace
\_\_\_\_\_\_\_\_\_\_\_\_\_\_\_\_\_\_\_\_\_\_\_ & \_\_\_\\
Total & 100\\
\bottomrule
\end{tabular}
\centering
\begin{tabular}[t]{llllll}
\toprule
Grade & Range & Grade & Range & Grade & Range\\
\midrule
NA & NA & A & >92\% & A- & 90\%-92\%\\
B+ & 87\%-89\% & B & 83\%-86\% & B- & 80\%-82\%\\
C+ & 77\%-79\% & C & 73\%-76\% & C- & 70\%-72\%\\
D+ & 67\%-69\% & D & 63\%-66\% & D- & 60\%-62\%\\
F & <60\% & NA & NA & NA & NA\\
\bottomrule
\end{tabular}
\end{table}

\hypertarget{course-policies}{%
\section{Course Policies}\label{course-policies}}

\hypertarget{student-responsibility-highlights}{%
\subsection{Student Responsibility
Highlights}\label{student-responsibility-highlights}}

\begin{itemize}
\tightlist
\item
  Students should log on to Canvas at least once a day to check for
  course updates. All links to external sites such as Zoom and office
  appointments, can be found on the course's Canvas homepage.
\item
  Student should review module readings and lectures before discussion
  sessions.
\item
  Students should contribute equally to group work and identify
  classmates who do not.
\item
  Students will not disrespect each others religious, cultural,
  political, or ideological beliefs.
\item
  Students will accept each others right to reasonably debate religious,
  cultural, political, or ideological beliefs of any kind.\footnote{Here,
    a reasonable argument must present at least one statement for and
    one against your chosen point of view.}
\end{itemize}

\hypertarget{rules-for-on-line-interaction}{%
\subsection{Rules for On-line
Interaction}\label{rules-for-on-line-interaction}}

\begin{itemize}
\tightlist
\item
  When appearing on camera, you should be fully dressed as if you were
  physically attending a meeting on campus.
\item
  \textbf{Remember: sarcasm is fine; joke within reason; and sexual
  harassment is real. Try to be aware and understanding of the fact that
  our on-line communications might not be conveyed as positively as our
  in-class interactions might.}
\end{itemize}

\hypertarget{syllabus-change-policy}{%
\subsection{Syllabus Change Policy}\label{syllabus-change-policy}}

Except for changes that substantially affect implementation of the
evaluation (grading) statement, this syllabus is a guide for the course
and is subject to change with advance notice.

\hypertarget{university-policies}{%
\section{University Policies}\label{university-policies}}

\hypertarget{academic-honor-policy}{%
\subsection{Academic Honor Policy}\label{academic-honor-policy}}

The Florida State University Academic Honor Policy outlines the
University's expectations for the integrity of students' academic work,
the procedures for resolving alleged violations of those expectations,
and the rights and responsibilities of students and faculty members
throughout the process. Students are responsible for reading the
Academic Honor Policy and for living up to their pledge to ``\ldots be
honest and truthful and\ldots{} {[}to{]} strive for personal and
institutional integrity at Florida State University.'' (For more details
see the FSU Academic Honor Policy and procedures for addressing alleged
violations Links to an external site..)

\hypertarget{americans-with-disabilities-act-ada}{%
\subsection{Americans With Disabilities Act
(ADA)}\label{americans-with-disabilities-act-ada}}

Students with disabilities needing academic accommodation should (1)
register with and provide documentation to the Student Disability
Resource Center and (2) bring a letter to the instructor indicating the
need for accommodation and what type. Please note that instructors are
not allowed to provide classroom accommodation to a student until
appropriate verification from the Student Disability Resource Center has
been provided. This syllabus and other class materials are available in
alternative format upon request. For more information about services
available to FSU students with disabilities, contact the Student
Disability Resource Center (SDRC) 108 Student Services Building
\textbar{} Florida State University \textbar{} (850) 644-9566 (voice)
\textbar{} (850) 644-8504 (TDD) \textbar{} Email:
\href{mailto:sdrc@admin.fsu.edu}{\nolinkurl{sdrc@admin.fsu.edu}}

\hypertarget{notification-of-accommodation}{%
\subsection{Notification of
Accommodation}\label{notification-of-accommodation}}

Religious and ADA accommodations must be secured within the first 3
weeks of the semester. You cannot request accommodations on an
assignment after that assignment has become due. Accommodations will not
be discussed with other students.


\end{document}
